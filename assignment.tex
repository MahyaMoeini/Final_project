\documentclass[a4paper,12pt]{article}
\usepackage{amsmath,amssymb}
\usepackage{hyperref} 


\title{\textbf{Final Assignment: Integration of Tools and Practices}}
\author{Mahya Moeini\\402412453}
\date{\today}

\begin{document}


\maketitle


\\\textbf{Introduction}
\\
\\This document provides answers to the assignment questions.

\clearpage


\tableofcontents
\clearpage


\section{Git and GitHub}

\subsection{Repository Initialization and Commits}
Write about how you set up the repository for this assignment. Explain every step in detail.
\\
\\\textbf{1. Create a New Repository on GitHub}
\\Log in to your GitHub account at github.com.
Click the green New button (or navigate to the "Repositories" tab and select "New").
\\Fill in the repository details.
\\Public the Repository.
\\Initialize Repository: Check the box to add a README.md file to start the repository.
\\Click Create Repository.
\\
\\\textbf{2. Clone the Repository Locally}
\\Navigate to the repository on GitHub.
\\Copy the repository’s URL by clicking the Code button and selecting HTTPS or SSH.
\\Open a terminal (or Git Bash) on your local machine and navigate to the directory where you want to clone the repository.
\\git clone repositoryname
\\
\\\textbf{3. Add Your LaTeX Files}
\\Create your .tex file.
\\
\\\textbf{4. Track Changes with Git}
\\Use git status to check the current state of your repository.
\\Stage the changes with git add .
\\Commit the changes git commit -m "Initial commit: Add LaTeX files for assignment"
\\Push your committed changes to the GitHub repository  git push origin main
\\
\\
\\
\\
\subsection{GitHub Actions for LaTeX Compilation}
Provide a walkthrough of setting up GitHub Actions to automatically compile your LaTeX document and any challenges you encountered.
\\
\\\textbf{1.Set Up GitHub Actions Workflow}
\\GitHub Actions uses YAML files to define workflows. These YAML files are typically placed in the .github/workflows/ directory of your repository.
\\To create the workflow for LaTeX compilation, follow these steps:
\\Navigate to Your Repository: Open the GitHub repository where your LaTeX project is stored.
\\Create a New Directory for Workflows: Inside your repository, create a .github/workflows/ directory if it doesn’t already exist.
\\Create the Workflow File: Create a new YAML file (e.g., latex.yml) in the .github/workflows/ directory.
\\
\\\textbf{2. Write the Workflow YAML}
\\I use the template that was in your github.
\\
\\\textbf{3.Commit the Workflow File}
\\Once the latex.yml file is created, add and commit the file to your repository:  
\\git add .github/workflows/latex.yml
\\git commit -m "Add GitHub Actions workflow for LaTeX compilation"
\\git push origin main
\\After pushing the changes to GitHub, navigate to the Actions tab in your GitHub repository.
You should see the workflow listed, and it will trigger whenever you push to the main branch or create a pull request to main.
Check the logs of the workflow to ensure the LaTeX compilation process works correctly.
\\
\\\textbf{4.Accessing the Compiled PDF}
\\Once the workflow completes, you can download the compiled PDF from the Actions tab by selecting the relevant run and downloading the artifact (finalassignment.pdf).

\clearpage
\section{Exploration Task}
\subsection{Vim Advanced Features}
Explore and document 3 advanced features of Vim that were not covered in class.
\\\textbf{1. Macros: Automating Repetitive Tasks}
\\Vim allows you to record a sequence of actions into a macro, which can be replayed whenever needed. This is useful for automating repetitive editing tasks.
\\How to Record a Macro:
Press q followed by a letter to start recording. The letter will be the register where the macro is stored (e.g., qa will store the macro in register a).
Perform the actions you want to automate. Vim will record each keypress as part of the macro.
Press q again to stop recording.
\\How to Play Back a Macro:
To replay the macro, press @ followed by the register (e.g., @a will replay the macro stored in register a).
You can replay the macro multiple times by specifying a count before pressing @. For example, 5@a will play the macro 5 times.
\\Example Use Case:
If you're editing a document and need to add a comment to multiple lines, you can record a macro to perform the process (e.g., moving to the beginning of a line, typing #, and moving to the next line) and then replay it across all lines.
\\
\\\textbf{2. Buffers and Windows: Managing Multiple Files}
\\In Vim, buffers are in-memory representations of files, and windows are the views that display those buffers. You can easily switch between buffers, split windows, and work on multiple files simultaneously.
\\To list open buffers : :ls
\\To switch to a specific buffer, use:  :b buffernumber
\\To split the window horizontally:  :split
\\To split the window vertically:  :vsplit
\\You can navigate between split windows using Ctrl-w followed by a direction key:
\\Ctrl-w h to move to the left window
\\Ctrl-w j to move to the window below
\\Ctrl-w k to move to the window above
\\Ctrl-w l to move to the right window
\\Example Use Case:
You can open a file in one window, split the window, and then open a different file in the second window. This makes it easier to compare or edit multiple files at once without switching back and forth between tabs.
\\\textbf{3. Search and Replace with Regular Expressions}
\\Vim supports powerful search and replace using regular expressions:
\\Search with /pattern and navigate matches with n (next) or N (previous).
Replace with :
\texttt{\%s/old/new/g for global replacement or \%s/old/new/gc to confirm each one}





\end{document}

